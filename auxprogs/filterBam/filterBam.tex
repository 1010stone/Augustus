\documentclass[11pt]{article}
\usepackage{natbib}

% Default margins are too wide all the way around. I reset them here
\setlength{\topmargin}{-.5in}
\setlength{\textheight}{9in}
\setlength{\oddsidemargin}{.125in}
\setlength{\textwidth}{6.25in}
\setlength{\topmargin}{-.5in}
\setlength{\textheight}{9in}
\setlength{\oddsidemargin}{.125in}
\setlength{\textwidth}{6.25in} 


\title{Filtering read alignments in BAM format} 
\author{Tonatiuh Pe\~{n}a-Centeno \\
University of Greifswald}
\date{\today}

 
\begin{document}  
\maketitle
\abstract{This note describes the operation of filterBam, a C++ program that makes use of Bamtools API in 
order to filter alignment files stored in BAM format.}

\section{Main features}
filterBam is a C++ code that cleans alignment files in BAM format that is based on filterPSL, a Perl routine 
written by Prof. Mario Stanke for PSL files. filterBAM includes the following filtering options:

\begin{itemize}
	\item	Screens out unmapped alignments.
	\item	Screens out alignments that do not comply with a pre-defined coverage level (default=80).
	\item	Screens out alignments that do not comply with a pre-defined percentage of identity (default=).
	\item	Screeens out alignments whose insert gaps do not comply with a pre-defined distance (default=).
	\item	Filters in two modalities: single and paired-end reads.	
	\item	When in paired-read mode, writes to file a prospective list of common target genes.
	\item   When in paired-read mode, writes to file a prospective list of pairedness coverage. 
\end{itemize}

\section{Single reads}
filterBam has been 


\section{NOTES:}  
This document makes reference to the SAM/BAM format specification of \citet{heng09:SAM}.

\section{Bamtools}
Bamtools is a C++ wrapper API of the more well-known Samtools software. The latest version of Bamools 
is 2.0 and is available on the website 
	\begin{center}
		https://github.com/pezmaster31/bamtools/downloads
	\end{center}
\item 

\section{Test data}
We have generated a 

\section{Compilation}
Make sure to link with the ``-lz'' and ``-libbamtools.a'' flags on; where -lz refers to the ZLIB library, 
and libbamtools.a to the static bamtools library included in the software distribution. An example of 
how to compile and link follows: 

\begin{enumerate}
\begin{flushleft}
	\item	
		g++ -I\textbf{\$BAMTOOLS}/include   -g   -std=c++0x  -c filterBam.cc -o filterBam.o \\
	\item	g++     -g -std=c++0x  filterBam.o -o filterBam \textbf{\$BAMTOOLS}/lib/libbamtools.a -lz  
\end{flushleft}
\end{enumerate}
\vphantom{Nothing}
where \textbf{\$BAMTOOLS} is the path where Bamtools was installed.

Note that the flag ``-std=c++0'' has been used given that some of the functionalities of the filter require 
some of the newest features of GNU's g++ compiler. This and future versions of the software have been tested 
on Ubuntu's g++ version 4.4.3.

\section{How to run}
A run that will let pass most, if not all, readings: 
\begin{flushleft}
./filterBam input.bam output.bam --minCover 0 --minId 0  --insertLimit 10000000 --nointrons
\end{flushleft}
\textbf{Note:} that all options are provided at the very end.

\section{Coverage, percent of identity and insert length}
The coverage is computed as the sum of the alignment matches (sequence matches or mismatches) and 
the insertions to the reference. Both figures, alignment matches and insertions to the reference, correspond 
to CIGAR string operations $M$ and $I$, respectively. Thus the following is done 

\begin{equation}
	\mathrm{coverage} = \frac{\sum\mathrm{CIGAR}\left(M,I\right)}{qLength}
\end{equation}

An approximation to the percentage of identity is given by computing the query length and subtracting the 
so-called edit distance to the reference (tag ``NM'' in SAM jargon), i.e.

\begin{equation}
	\mathrm{percId} = \frac{qLength - \mathrm{Tag}(NM)}{qLength}
\end{equation}

The length of inserts is estimated by summing CIGAR operations ``M'' and ``I'', which correspond to alingment 
matches and deletions from the reference. In other words, we do the following

\begin{equation}
	\mathrm{InsertSize} = \frac{\sum\mathrm{CIGAR}\left(D,I\right)}{qLength}
\end{equation}


\bibliographystyle{abbrvnat}
\bibliography{bioinformatics}

\end{document}
